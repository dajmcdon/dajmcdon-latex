\documentclass[12pt]{article}
\usepackage[commenters={OA,AA,DJM}]{shortex} % adjust initials for comments
\usepackage{authblk}
\usepackage[round]{natbib}
\usepackage[margin=2.5cm]{geometry}
\newcommand{\email}[1]{\href{mailto:#1}{#1}}

% minor adjustments to ShorTeX
\let\argmin\relax\DeclareMathOperator*{\argmin}{argmin}
\let\argmax\relax\DeclareMathOperator*{\argmax}{argmax}
\renewcommand{\top}{\mathsf{T}}
\renewcommand{\d}{\mathsf{d}}

\graphicspath{{fig/}}

% -- Begin document --------------------------------------------------------

\title{Creative title: Where we give results about things}
\author[a,1]{Other Author}
\author[b]{Another Author}
\author[a]{Daniel J.\ McDonald}

\affil[a]{Department of Statistics, The University of British Columbia}
\affil[b]{Department of Music, Indiana University}
\date{Last updated: \today}

\begin{document}
\maketitle
%\RaggedRight % uncomment to enable ragged right

\babs
Write an abstract.

\vspace{10pt}
\noindent Keywords: fractals $|$ quavers $|$ carbonara
\eabs

\footnotetext[1]{To whom correspondence should be
  addressed. E-mail: \email{some.email@stat.ubc.ca}} 


\section{Introduction}\label{sec:intro}

We write some math for fun:

\[
\label{eq:1}
\int_{-\infty}^{\infty} \frac{1}{\sqrt{2\pi}|\Sigma|^{n/2}}
\exp\cbra*{-\frac{1}{2}(y - \mu)\T \Sigma^{-1} (y-\mu)} \d y = 1.
\]

We encourage the use of the various macros defined in
\href{https://github.com/trevorcampbell/shortex/}{ShorTeX},
so do your best. It makes things easier to read, but also provides lots of
necessary mathematics definitions that render nicely. 

Be careful with things like KL divergence and conditional probability
statements. I find that 
\[\KL(q\ \Vert\ p)\]
looks much better than 
\[\KL(q||p),\] 
and I similarly prefer 
\[
Y\ |\ X \sim \Norm(X,\ \sigma^2) 
\quad\textrm{to}\quad Y | X \sim \Norm(X,\sigma^2).
\] 
Note that the reals are $\reals^p$. There is also
\[
\shbeta = \argmin_{\beta \in \reals^p} \frac{1}{2n}\norm{\sky - \skX\beta}_2^2 + 
\lambda \norm{\beta}_1. 
\]


\subsection{Cleveref}\label{sec:clever}

We prefer to use cleveref to get nice references to things. For example, you can
say that \cref{eq:1} was printed in \cref{sec:intro}. No need to write out
things like ``Section''.

\section{Some best practices}

Some of these are taken from
\href{http://faculty.marshall.usc.edu/Jacob-Bien/papers/manuscript-checklist.pdf}{Jacob
Bien}. Note the use of the ShorTeX itemize environment style below.

\bitem
\item Section titles should be all title case or all sentence case. Don't mix and match.
\item I prefer data set to dataset.
\item I prefer data to be singular. There remains debate on this point. When you
use the word ``datum'' in a sentence, then we can argue. Data is a mass noun,
like ``information''. We don't say ``How many data are enough?'', we say ``How much
data is enough.'' Enough said.  
\item Terminology is lower case, unless it's a person's name: \Nystrom extension
and lasso.
\item Equations are parts of the sentence. Displayed equations almost always
have a comma or period after. Very rarely is there a colon or comma \emph{before}
a displayed equation.
\item Don't start sentences with math (``$\Sigma$ is the covariance of
$\skX$.'') or the name of a software package that's lowercase, \eg
``\texttt{glmnet} is my favourite software''.
\item Don't use contractions.
\item No need to put dollar signs around numbers: 12 versus $12$.
\item DO put dollar signs around math: $p$ not p.
\item Use $x\gg y$ not $x >>y$.
\item Careful with parentheticals and references. Wrong: (see, \eg
\citet{Akaike1973}). Right: (see, \eg \citealp{Akaike1973}).
\item Never use \texttt{eqnarray}, always use \texttt{align}. Note that ShorTeX
makes \verb+\[   \]+ into an align environment, so you can just use that always.
\item For editing purposes, it is much better if the text is hard-wrapped rather
than soft wrapped.
\eitem

\subsection{Tables}

\Cref{tab:1} is a nice looking table. Strive for these.

\btab
\bcent
\btabr{@{}lr@{}}
\toprule
Ingredient & Quantity\\
\midrule
Fusili & 100 g\\
Eggs & 2\\
Salt & 1 tsp\\
Guanciale & 50 g\\
Pepper & \nicefrac{1}{2} tsp\\
Grated parmesan & \nicefrac{1}{4} c\\
\bottomrule
\etabr
\caption{This is a nice looking table. It might make carbonara.}
\label{tab:1}
\ecent
\etab

\section{Discussion} 

We made amazing contributions to the world of musical fractal pasta 
\citep{McDonald2017,Tibshirani2013}. We use Natbib, so be sure to use
\citep{Stein1981} for parenthetical references. Or you can say, according to
\citet{HastieTibshirani2009}, we should strive to balance truth and lies.

\bibliographystyle{rss}
\bibliography{dajmcdon}      

\end{document}